\documentclass[a4paper, oneside, final]{scrartcl}

\usepackage[utf8x]{inputenc}
\usepackage[brazil]{babel}
\usepackage{amssymb}

\usepackage{soul}
\usepackage{scrpage2}
\usepackage{titlesec}
\usepackage{marvosym}
\usepackage{tabularx}

\usepackage[hmargin=2cm,vmargin=3.5cm,noheadfoot]{geometry}

\titleformat{\section}{\large\scshape\raggedright}{}{0em}{}[\titlerule]
\pagestyle{scrheadings}
\renewcommand{\headfont}{\normalfont\rmfamily\scshape}

% add the symbols for email and phone contact data
\cofoot{
%\so{Av. Xxxxxxxxx xx Xxxxx, YYYY - São Paulo, SP}\\
\so{ {\Large\Letter} davidrobert@gmail.com \ {\Large\Telefon} (11) 9812-6455}
}

\begin{document}

\begin{center}
\textsc{\Huge{\so{David Robert C. Campos}}}\\ \ \\
%\textsc{\Huge{\so{David Robert Camargo Campos}}}\\ \ \\

\section{Objetivo}
Atuar na área de tecnologia da informação como líder de equipe e/ou engenheiro de software.

\section{Resumo}

\begin{tabularx}{0.97\linewidth}{X}
  Engenheiro de software com mais de 9 anos de experiência em programação com grande entusiasmo por novos desafios. \\ \ \\
  Excelentes conhecimentos em diferentes linguagens de programação e tecnologias, incluindo C/C++, Java, Python, Perl e SQL.
\end{tabularx}

\section{Formação Acadêmica}

\begin{tabularx}{0.97\linewidth}{p{2cm}X}
2004-2008   & Mestrado em Ciências da Computação\\
            & Universidade de São Paulo, USP\\
            & Título: Reparo de Plano por Refinamento Reverso\\
            & Área:  Inteligência Artificial\\ \\

1999-2003   & Graduação em Ciência da Computação\\
            & Pontifícia Universidade Católica de São Paulo, PUC-SP\\ \\

2000-2001   & Aspirante à Oficial de Comunicações\\
            & Centro de Preparação de Oficiais da Reserva de São Paulo, CPOR-SP
\end{tabularx}

\section{Formação Complementar}

\begin{tabularx}{0.97\linewidth}{p{2cm}X}
2004        & Introdução à Computação Científica (60h)\\
            & Universidade de São Paulo, USP\\ \\

2003        & Tópicos de Programação (60h)\\
            & Universidade de São Paulo, USP
\end{tabularx}

\section{Prêmios e Certificações}

\begin{tabularx}{0.97\linewidth}{p{2cm}X}
2010        & SCJP --- Sun Certified Java Programmer\\ \\
2000        & Medalha Correia Lima --- Prêmio recebido por terminar o curso no CPOR em primeiro lugar
\end{tabularx}

\section{Atuação Profissional}

\begin{tabularx}{0.97\linewidth}{p{2cm}X}
Periodo     & 2006 --- Atual\\
Empresa     & Registro.br\\
Cargo       & Engenheiro de Software\\
Atividades  & Desenvolvimento de aplicações em Java, C/C++ e Python em ambientes Linux e FreeBSD. Participação ativa no desenvolvimento do servidor DNS do NIC.br, incluindo as extensões de segurança DNSSEC. Apresentação de palestras e tutoriais nas áreas de DNS e segurança da informação, representando a companhia em diversos eventos de tecnologia.\\
% A partir de 2007 passou a ministrar cursos de segurnaça voltado para públicos específicos, como provedores de acesso, bancos e governo.\\
            & \ \\

Periodo     & 2010 --- Atual\\
Empresa     & FATEC-SCS $-$ Faculdade de Tecnologia de São Caetano do Sul\\
Cargo       & Professor\\
Atividades  & Professor da disciplina Avaliação de Desenpenho de Sistemas Computacionais para os cursos de graduação Segurança da Informação e Jogos Digitais.\\
            & \ \\

Periodo     & 2008 --- 2010\\
Empresa     & Universidade Anhanguera\\
Cargo       & Professor\\
Atividades  & Professor das disciplinas Sistema de Banco de Dados, Programação em Banco de Dados, Gestão de Qualidade e Lógica de Programação para cursos de graduação e pós-graduação.\\
            & \ \\

Periodo     & 2009 --- 2010\\
Empresa     & UNIBAN $-$ Universidade Bandeirante de São Paulo\\
Cargo       & Professor\\
Atividades  & Professor das disciplinas Estrutura de Dados e Desenvolvimento Web Avançado para cursos de graduação Sistemas de Informação.\\
            & \ \\

Periodo     & 2004 --- 2008\\
Empresa     & FATED $-$ Faculdade de Tecnologia Diamante\\
Cargo       & Professor\\
Atividades  & Professor das disciplinas Introdução a Programação, Estrutura de Dados e Sistema de Banco de Dados para cursos de graduação.\\
            & \ \\

Periodo     & 2002 --- 2006\\
Empresa     & Agência Estado\\
Cargo       & Analista de Sistemas\\
Atividades  & Desenvolvimento de servidores de aplicações para distribuição de informação em tempo real, aplicações web de grande porte e ferramentas internas diversas. Principais tecnologias utilizadas: Perl, PHP, C++, XHTM, DHTML e JavaScript.\\
%            & \ \\
\end{tabularx}

\section{Idiomas}

\begin{tabularx}{0.97\linewidth}{p{2cm}X}
Inglês      & Técnico\\
%Italiano    & Básico
\end{tabularx}

\section{Participação em eventos (como palestrante)}

\begin{tabularx}{0.97\linewidth}{p{2cm}X}
2010        & IV Congresso de Segurança da Informação e Comunicações para Go\-ver\-no\\
            & Título da Apresentação: DNSSEC e Segurança da Informação\\ \\

2010        & 4º PTT Fórum - Encontro dos Sistemas Autônomos da Internet no Brasil\\
            & Título da Apresentação: DNS e DNSSEC\\ \\

2010        & Semana da Segurança da Informação --- FATEC-SCS\\
            & Título da Apresentação: DNS, DNSSEC e Boas Práticas de Segurança\\ \\

2010        & XII Semana Integrada --- PUC-Campinas\\
            & Título da Apresentação: DNSSEC e vulnerabilidade dos DNSs\\ \\

2009        & 27º Reunião do Grupo de Trabalho de Engenharia e Operações de Redes\\
            & Título da Apresentação: DNSSHIM - DNSSEC automatizado\\ \\

2009        & 3º PTT Fórum - Encontro dos Sistemas Autônomos da Internet no Brasil\\
            & Título da Apresentação: Tutorial de DNS\\ \\

2009        & Campus Party 2009\\
            & Título da Apresentação: DNSSEC o que é e porquê precisamos dele\\ \\

2008        & Campus Party 2008\\
            & Título da Apresentação: Tutorial de DNSSEC\\ \\

2008        & Slackware Show 2008\\
            & Título da Apresentação: Tutorial de DNSSEC\\ \\

2007        & 23ª Reunião do Grupo de Trabalho de Engenharia e Operações de Redes\\
            & Título da Apresentação: Tutorial de DNSSEC
\end{tabularx}

%\section{Skills}
%\section{Publications}
%\section{Languages}
%\section{Interests}

\end{center}

\end{document}
